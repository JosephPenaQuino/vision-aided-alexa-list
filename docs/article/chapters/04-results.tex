\section{Results}

This section presents the results obtained during the model's training process, where key metrics such as accuracy, precision, and recall were evaluated across three distinct scenarios: basic, intermediate, and advanced. Additionally, the implementation cost of AWS services is analyzed.


\subsection{Model metrics}

\subsubsection{Train model}

The model was evaluated using various labels, with key metrics such as F1 score, precision, recall, and an assumed threshold considered during the training phase. For the label "club-social-purple," the model achieved an F1 score of 0.795, with a precision of 1.000 and a recall of 0.660, using an assumed threshold of 0.600. Similarly, the label "club-social-red" resulted in an identical F1 score and precision but with a slightly lower assumed threshold of 0.550. The "club-social-yellow" label showed improved performance with an F1 score of 0.923, precision of 0.857, and a perfect recall of 1.000, under a threshold of 0.488. For the labels "lacta" and "piraque," the model achieved perfect scores across all metrics (F1 score, precision, recall of 1.000), with thresholds of 0.605 and 0.803, respectively.

\subsubsection{Basic scenario}

In the basic scenario, the model was evaluated using a straightforward dataset, where it achieved an accuracy of 86\%. This scenario serves as a benchmark for the model's baseline performance under minimal complexity.

\subsubsection{Intermediate scenario}

In the intermediate scenario, a moderately complex dataset was used to test the model's ability to handle more nuanced inputs. The model demonstrated improved performance with an accuracy of 67\%, reflecting its capacity to generalize beyond simple patterns.

\subsubsection{Advanced scenario}

For the advanced scenario, the model was tested on a highly complex dataset that included challenging cases and edge conditions. The accuracy achieved in this scenario was 50\%, showcasing the model's robustness and scalability when faced with real-world complexities.

\subsection{Cost implementation}

The cost analysis for the system was performed by considering various AWS services used in the deployment and training phases. The initial cost totaled \$ 4.76, which covered SageMaker and Rekognition training expenses. SageMaker incurred a one-time cost of \$ 3.76 for processing 47 images, and Rekognition training required 30 minutes at \$ 2/hour, amounting to \$ 1. Other services, such as Lambda (for both the Alexa skill and computer vision), S3 storage, and DynamoDB, had no initial cost due to the minimal usage during the initial phase.

The daily operating cost was primarily driven by Rekognition inference, which incurred \$ 96 per day for continuous usage. Other services, such as Lambda, S3, and DynamoDB, had negligible or no significant daily costs, as the usage remained within free tiers or minimal limits.

In summary, the system’s total initial cost was \$ 4.76, with an ongoing daily cost of approximately \$ 96.

